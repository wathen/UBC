\documentclass[11pt]{article}

\usepackage{mathrsfs}
\usepackage{amsmath}
\usepackage{amsfonts}
\usepackage{graphicx}
\usepackage{nicefrac}
\usepackage{subfigure}
\usepackage{algorithm}
\usepackage{paralist}
\usepackage[geometry]{ifsym}
\usepackage{rotating}
\usepackage[framed,numbered,autolinebreaks,useliterate]{mcode}
\usepackage[normalem]{ulem}


\usepackage{fancyheadings}

\newcommand{\com}[1]{\texttt{#1}}
\newcommand{\DIV}{\ensuremath{\mathop{\mathbf{DIV}}}}
\newcommand{\GRAD}{\ensuremath{\mathop{\mathbf{GRAD}}}}
\newcommand{\CURL}{\ensuremath{\mathop{\mathbf{CURL}}}}
\newcommand{\CURLt}{\ensuremath{\mathop{\overline{\mathbf{CURL}}}}}
\newcommand{\nullspace}{\ensuremath{\mathop{\mathrm{null}}}}


\newcommand{\FrameboxA}[2][]{#2}
\newcommand{\Framebox}[1][]{\FrameboxA}
\newcommand{\Fbox}[1]{#1}

%\usepackage[round]{natbib}

\newcommand{\half}{\mbox{\small \(\frac{1}{2}\)}}
\newcommand{\hf}{{\frac 12}}
\newcommand {\HH}  { {\bf H} }
\newcommand{\hH}{\widehat{H}}
\newcommand{\hL}{\widehat{L}}
\newcommand{\bmath}[1]{\mbox{\bf #1}}
\newcommand{\hhat}[1]{\stackrel{\scriptstyle \wedge}{#1}}
\newcommand{\R}{{\rm I\!R}}
\newcommand {\D} {{\vec{D}}}
\newcommand {\sg}{{\hsigma}}
%\renewcommand{\vec}[1]{\ensuremath{\mathbf{#1}}}
\newcommand{\E}{\vec{E}}
\renewcommand{\H}{\vec{H}}
\newcommand{\J}{\vec{J}}
\newcommand{\dd}{d^{\rm obs}}
\newcommand{\F}{\vec{F}}
\newcommand{\C}{\vec{C}}
\newcommand{\s}{\vec{s}}
\newcommand{\N}{\vec{N}}
\newcommand{\M}{\vec{M}}
\newcommand{\A}{\vec{A}}
\newcommand{\B}{\vec{B}}
\newcommand{\w}{\vec{w}}
\newcommand{\nn}{\vec{n}}
\newcommand{\cA}{{\cal A}}
\newcommand{\cQ}{{\cal Q}}
\newcommand{\cR}{{\cal R}}
\newcommand{\cG}{{\cal G}}
\newcommand{\cW}{{\cal W}}
\newcommand{\hsig}{\hat \sigma}
\newcommand{\hJ}{\hat \J}
\newcommand{\hbeta}{\widehat \beta}
\newcommand{\lam}{\lambda}
\newcommand{\dt}{\delta t}
\newcommand{\kp}{\kappa}
\newcommand {\lag} { {\cal L}}
\newcommand{\zero}{\vec{0}}
\newcommand{\Hr}{H_{red}}
\newcommand{\Mr}{M_{red}}
\newcommand{\mr}{m_{ref}}
\newcommand{\thet}{\ensuremath{\mbox{\boldmath $\theta$}}}
\newcommand{\curl}{\ensuremath{\nabla\times\,}}
\renewcommand{\div}{\nabla\cdot\,}
\newcommand{\grad}{\ensuremath{\nabla}}
\newcommand{\dm}{\delta m}
\newcommand{\gradh}{\ensuremath{\nabla}_h}
\newcommand{\divh}{\nabla_h\cdot\,}
\newcommand{\curlh}{\ensuremath{\nabla_h\times\,}}
\newcommand{\curlht}{\ensuremath{\nabla_h^T\times\,}}
\newcommand{\Q}{\vec{Q}}
\renewcommand{\J}{\vec J}
\renewcommand{\J}{\vec J}

\newcommand{\me}{Maxwell's equations }

\newcommand{\partialt}[1]{\frac{\partial #1}{\partial t}}
\newcommand{\cref}[1]{(\ref{#1})}
% \newcommand{\Ct}{\ensuremath{C^{\mbox{\tiny{T}}}}
\newcommand{\Bt}{\ensuremath{B^{\mbox{\tiny{T}}}}}
\newcommand{\Ct}{\ensuremath{C^{\mbox{\tiny{T}}}}}
\newcommand{\Dt}{\ensuremath{D^{\mbox{\tiny{T}}}}}
% \renewcommand{\baselinestretch}{1.40}\normalsize
\usepackage{setspace}
\usepackage{amsthm}
\newtheorem{prop}{Proposition}[section]

\onehalfspacing
\begin{document}

$$
\begin{bmatrix}
F & \Bt & \Ct & 0 \\
B & 0 & 0 & 0 \\
-C & 0 & M & \Dt  \\
0 & 0 & D &0
\end{bmatrix}
\begin{bmatrix}
u \\ p \\ b \\ r
\end{bmatrix}
= \lambda
\begin{bmatrix}
F & \Bt & \Ct & 0 \\
0 & -M_s & 0 & 0 \\
-C & 0 & M+\Dt L^{-1} D & 0  \\
0 & 0 & 0 &L
\end{bmatrix}
\begin{bmatrix}
u \\ p \\ b \\ r
\end{bmatrix}
$$
Writing this as four equations gives:
\begin{subequations} \label{system}
    \begin{align}
        (\lambda -1) (Fu+\Bt p+ \Ct b) &= 0\\
        B u &= -\lambda M_s p\\
        (\lambda - 1)C u +(1-\lambda)Mb +\Dt r -\lambda \Dt L^{-1} D b &=0\\
        Db &= \lambda Lr
    \end{align}
\end{subequations}
Substituting (\ref{system}d) into (\ref{system}c) we obtain
\begin{equation} \label{sub}
(\lambda - 1)C u +(1-\lambda)Mb + (\Dt -\lambda^2 \Dt)r =0.
\end{equation}
From (\ref{system}a) we can see that $\lambda = 1$ satisfies the equation. Substituting $\lambda = 1$ into the other three equations gives:
\begin{subequations}
    \begin{align}
        B u &= - M_s p\\
         \Dt r - \Dt L^{-1} D b &=0\\
        Db &=  Lr
    \end{align}
\end{subequations}
From (\ref{sub}) we then obtain the following eigenvector $(u, -M_s^{-1}Bu,b,L^{-1}Db)$. I can't see why $u\neq0$ or $b \neq0$ to get the exact statement which you asked me to show. I coded up the eigenvalue problem with random matrices and it seemed to show that there where $n+\hat{n}+4$ eigenvalues of $\lambda = 1$.

Consider $\lambda \neq 1$ then (\ref{system}) is
\begin{subequations} \label{system1}
    \begin{align}
        Fu+\Bt p+ \Ct b &= 0\\
        B u &= -\lambda M_s p\\
        (\lambda - 1)C u +(1-\lambda)Mb +\Dt r -\lambda \Dt L^{-1} D b &=0\\
        Db &= \lambda Lr
    \end{align}
\end{subequations}
Substitute (\ref{system1}d) into (\ref{system1}c) and with some simplification gives
\begin{equation} \label{inv}
-\lambda Cu+ (\lambda Mu +(1+\lambda)\Dt L^{-1}D)b=0.
\end{equation}
Let $A = \lambda Mu +(1+\lambda)\Dt L^{-1}D$, then since $M$ is positive semi-definite and $\Dt L^{-1}D$ is positive definite then $A$ is non-singular, hence, $b = \lambda A^{-1} Cu$. Using this expression for $b$ and $p = - \frac{1}{\lambda} M_s^{-1}Bu$ with (\ref{system1}a) to eliminate $b$ and $p$ gives
$$(F-\frac{1}{\lambda} \Bt M_s^{-1} B+\lambda \Ct A^{-1} C)u = 0 \implies \mathcal{A}u = 0.$$
Since $\mathcal{A}$ is invertible then $u=0$, hence $p=0$. Looking at the case when $\lambda = -1$ then (\ref{inv}) $$Cu = Mb,$$ since $M$ is singular then $u=0$, hence $p=0$. Therefore, the eigenvector associated with $\lambda = -1$ is $(0,0,b, -L^{-1}Db)$.

When considering $\lambda \neq 1$ or $-1$ then (\ref{inv}) reduces to
$$ (\lambda Mu +(1+\lambda)\Dt L^{-1}D)b=0$$
which means that $b\in \mbox{ker}(\lambda Mu +(1+\lambda)\Dt L^{-1}D)$ since $u=p=0$.

I tried to download and install deall.II but if said that I needed to install loads of other packages first which I am currently installing now.

% Substituting $\lambda = -1$ into (\ref{system}) gives
% \begin{subequations}
%     \begin{align}
%         Fu+\Bt p+ \Ct b &= 0\\
%         B u &= M_s p\\
%         -2 C u +2Mb +\Dt r +\Dt L^{-1} D b &=0\\
%         Db &= -Lr
%     \end{align}
% \end{subequations}
% Since both $\Bt$ and $\Ct$ have a null space then $u=0$, hence $p = 0$. Therefore we have the following eigenvector $(0,0,b, -L^{-1}Db)$ which corresponds to the eigenvalue $\lambda = -1$.


$\mathscr{i}$




\end{document}
