\documentclass{article}
\usepackage{afterpage}
\usepackage{float}
\usepackage{longtable}
\usepackage{graphicx}
\usepackage{pdflscape}
\usepackage[numbers,sort&compress]{natbib}
\usepackage{psfrag}

\usepackage{amsmath}
\usepackage{amsfonts}
\usepackage{graphicx}
%\usepackage{nicefrac}
\usepackage{graphicx}
\usepackage{caption}
% \usepackage{subcaption}
\usepackage{subfigure}
% \usepackage{algorithm}
% \usepackage{paralist}
% % \usepackage[geometry]{ifsym}
\usepackage{rotating}
\usepackage{setspace}
\newcommand{\uu}[1]{\boldsymbol #1}
\usepackage{listings}
\usepackage{xcolor}
\lstset{language=C++,
                keywordstyle=\color{blue},
                stringstyle=\color{red},
                commentstyle=\color{green},
                morecomment=[l][\color{magenta}]{\#}
}
\doublespace
\begin{document}


\section{$\kappa$ precondition test}

Consider a discretised incompressible Magnetohydrodynamics problem of the following form
\begin{equation} \label{eq:K}
    {\mathcal K}_{\rm MH} = \left(
\begin{array}{cccc}
A+O & B^T & C^T & 0\\
B & 0 & 0 & 0\\
-C & 0 & M & D^T \\
0 & 0 & D & 0
\end{array}
\right).
\end{equation}
Using the inner-outer precondition 
\begin{equation}
\label{eq:mhd_pc_ls}
\mathcal{K}_{\rm MH} =
\left(
\begin{array}{cccc}
F & B^T & C^T & 0\\
0 & -{S} & 0 & 0 \\
-C & 0 & M+X & 0\\
0 & 0 & 0 & L
\end{array}
\right),
\end{equation}
from [Dan's thesis] and [Michael's thesis] as a starting point we re-oder the variables to $(u,b,p,r)$ to yield
\begin{equation}
\label{eq:mhd_pc_ls}
\mathcal{M}_{\rm MH} =
\left(
\begin{array}{cccc}
F & C^T & B^T & 0\\
-C & M+X & 0 & 0 \\
0 & 0 & -S & 0\\
0 & 0 & 0 & L
\end{array}
\right).
\end{equation}
Using the preconditioner $\mathcal{M}_{\rm MHD}$ depends on an efficient way to implement
\begin{equation}
\label{eq:mhd_pc_ls}
\mathcal{M}_{\rm outer}=\left(
\begin{array}{cc}
F & C^T \\
-C & M+X 
\end{array}
\right)^{-1}.
\end{equation}
The approach used in [Dan's thesis] and [Michael's thesis] was to approximate $\mathcal{M}_{\rm outer}$ with an "inner" GMRES solve taking the block diagonal matrices as the preconditioner. However, this inner solver limits the efficiency of the preconditioning approach, therefore consider different possible approximations of $\mathcal{M}_{\rm outer}$ may increase the performance of this approach. Performing block LDL decomposition gives two possible preconditions of the following form
\begin{equation}
    \mathcal{M}_{1} = 
    \begin{pmatrix}
        F & 0 \\
        0 & M+X + CF^{-1}C^{T}
    \end{pmatrix}, \quad
    \mathcal{M}_{2} = 
    \begin{pmatrix}
        F+C^T(M+X)^{-1}C & 0 \\
        0 & M+X
    \end{pmatrix}.
\end{equation}
It has been show that using $\mathcal{M}_1$ or $\mathcal{M}_2$ as a preconditioner for a Kyrlov subspace method yields exactly 3 eigenvalues [Murphy, Golub, Wathen]. Therefore, $\mathcal{M}_1^{-1}$ or $\mathcal{M}_2^{-1}$ may yield possible alternative and better approximations to $\mathcal{M}_{\rm outer}$.

\subsection{Approximation of $C^TA^{-1}C$ and $C\tilde{A}^{-1}C^T$ where $A$ and $\tilde{A}$ are Laplacians}

The operator $\tilde{\mathcal{F}}u$ is given by 
\begin{equation}
\begin{array}{cl}
C\tilde{A}^{-1}C^T(u) &= -b \times \{ \nabla \times \Delta^{-1} [\nabla \times (u\times b)]\}, \\
&= -b\times \nabla\times \Delta^{-1} \nabla \times (u\times b).
\end{array}
\end{equation}
The second operator $\mathcal{F}b$ is given by
\begin{equation}
\begin{array}{cl}
C^t{A}^{-1}C(b) &= \nabla \times ( [\Delta^{-1}\{ -b\times(\nabla \times b)\}] \times b),\\
&= -\nabla \times ([\frac{1}{2}\Delta^{-1}\nabla (b\cdot b)]\times b - [\Delta^{-1}(b\cdot\nabla)b]\times b),\\
&= -\frac{1}{2}\nabla \times \Delta^{-1}\nabla (b\cdot b)\times b - \nabla \times \Delta^{-1}(b\cdot\nabla)b)\times b,\\
&= \nabla \times \Delta^{-1}(-\frac{1}{2}\nabla (b\cdot b)\times b -(b\cdot\nabla)b)\times b),\\
&= [\nabla \times \Delta^{-1}(\nabla \times b)\times b]\times b.
\end{array}
\end{equation}

The efficiency of the two approaches $\mathcal{F}b$ and $\tilde{\mathcal{F}}u$ rely on a good approximation of $\nabla \times \Delta^{-1}\nabla$. In the resent technical report from [Phillips et. al] they showed that 
$$ \nabla \times \Delta^{-1}\nabla c = -c.$$ 
Using this identity the two operators simplify to:
$$\tilde{\mathcal{F}}u = b\times(u\times b), \quad \mathcal{F}b = (b\times b)\times b.$$
Here we notice that for the simplification of $\mathcal{F}b$ has an $b\times b$ term which is identically zero. However, since the equations are linearised then 
$$\mathcal{F}b = (b^k\times b)\times b^k.$$
In 2-dimensions, let $b^k = (b_1^k,b_2^k,0)$ and $b = (b_1,b_2,0)$ then 
$$
(b^k \times b)\times b^k = 
\begin{pmatrix}
    (b_2^k)^2 & -b_1^kb_2^k\\
    -b_1^kb_2^k & (b_1^k)^2
\end{pmatrix}
\begin{pmatrix}
    b_1\\b_2
\end{pmatrix}.
$$
Using $u = (u_1,u_2,0)$ and linearising around the two magnetic terms in $\tilde{\mathcal{F}}u$ yields
$$
b_k \times (u\times b_k) = 
\begin{pmatrix}
    (b_2^k)^2 & -b_1^kb_2^k\\
    -b_1^kb_2^k & (b_1^k)^2
\end{pmatrix}
\begin{pmatrix}
    u_1\\u_2
\end{pmatrix}
$$



\subsection{Numerical testing}

Construction of the precondition
\begin{equation}
    \begin{pmatrix}
        F & C^T & B^T & 0 \\
        0 & M+X+CF^{-1}C^T& 0 & 0 \\
        0 & 0 & S & 0 \\
        0 & 0 & 0 & L
    \end{pmatrix},
\end{equation}
produces the following results
\begin{table}[h!]
\begin{tabular}{lrrrrrll}
\hline
 l &   DoF &  AV solve Time &  Total picard time &  picard iterations & Av Outer its \\
\hline
 1 &    51 &       0.012178 &           0.455623 &                  5 &         11.6 \\
 2 &   163 &       0.034960 &           0.543420 &                  5 &         25.8 \\
 3 &   579 &       0.134840 &           1.109627 &                  5 &         33.0 \\
 4 &  2179 &       1.488171 &           8.177399 &                  5 &         35.2 \\
 5 &  8451 &      99.375478 &         498.770225 &                  5 &         36.6 \\
\hline
\end{tabular}
\caption{$\kappa = 1$, $\nu_m = 1$ and $\nu=1$}
\end{table}

\begin{table}[h!]
\begin{tabular}{lrrrrrll}
\hline
 l &   DoF &  AV solve Time &  Total picard time &  picard iterations & Av Outer its \\
\hline
1 &    51 &       0.011994 &           0.536929 &                  6 &         11.5\\
2 &   163 &       0.034794 &           0.649704 &                  6 &         26.3\\
3 &   579 &       0.133624 &           1.359779 &                  6 &         34.0\\
4 &  2179 &       1.463671 &           9.618563 &                  6 &         36.7\\
5 &  8451 &      83.979013 &         505.978769 &                  6 &         37.3\\
\hline
\end{tabular}
\caption{$\kappa = 10$, $\nu_m = 1$ and $\nu=1$}
\end{table}

\begin{table}[h!]
\begin{tabular}{lrrrrrll}
\hline
 l &   DoF &  AV solve Time &  Total picard time &  picard iterations & Av Outer its \\
\hline
 1 &    51 &       0.011866 &           0.616299 &                  7 &         10.9  \\
 2 &   163 &       0.034967 &           0.756720 &                  7 &         26.7  \\
 3 &   579 &       0.134432 &           1.582807 &                  7 &         34.1  \\
 4 &  2179 &       1.464692 &          11.227711 &                  7 &         36.4  \\
 5 &  8451 &      87.282682 &         613.476128 &                  7 &         37.6  \\
\hline
\end{tabular}
\caption{$\kappa = 100$, $\nu_m = 1$ and $\nu=1$}
\end{table}


Now consider the following preconditioner
\begin{equation}
    \begin{pmatrix}
        F - C^T(M+X)^(-1)C& C^T & B^T & 0 \\
        0 & M+X& 0 & 0 \\
        0 & 0 & S & 0 \\
        0 & 0 & 0 & L
    \end{pmatrix},
\end{equation}
Using this preconditioner yields:
\begin{table}[h!]
\begin{tabular}{lrrrrrll}
\hline
 l &   DoF &  AV solve Time &  Total picard time &  picard iterations & Av Outer its \\
\hline
1 &    51 &       0.012170 &           0.454375 &                  5 &         11.6\\
2 &   163 &       0.034755 &           0.541899 &                  5 &         25.8\\
3 &   579 &       0.134306 &           1.105293 &                  5 &         33.0\\
4 &  2179 &       1.484157 &           8.151595 &                  5 &         35.2\\
5 &  8451 &     117.987729 &         591.895518 &                  5 &         36.6\\
\hline
\end{tabular}
\caption{$\kappa = 1$, $\nu_m = 1$ and $\nu=1$}
\end{table}

\begin{table}[h!]
\begin{tabular}{lrrrrrll}
\hline
 l &   DoF &  AV solve Time &  Total picard time &  picard iterations & Av Outer its \\
\hline
 1 &    51 &       0.013092 &           3.153271 &                  6 &         11.5 \\
 2 &   163 &       0.036800 &           0.730741 &                  6 &         26.3 \\
 3 &   579 &       0.140951 &           1.440613 &                  6 &         34.0 \\
 4 &  2179 &       1.817301 &          11.879250 &                  6 &         36.7 \\
 5 &  8451 &     114.076549 &         686.811843 &                  6 &         37.3 \\
\hline
\end{tabular}
\caption{$\kappa = 10$, $\nu_m = 1$ and $\nu=1$}
\end{table}

\begin{table}[h!]
\begin{tabular}{lrrrrrll}
\hline
 l &   DoF &  AV solve Time &  Total picard time &  picard iterations & Av Outer its \\
\hline
 1 &    51 &       0.013156 &           3.384350 &                  7 &         10.9\\
 2 &   163 &       0.038609 &           0.911713 &                  7 &         26.7\\
 3 &   579 &       0.152467 &           1.862579 &                  7 &         34.1\\
 4 &  2179 &       2.115897 &          15.969938 &                  7 &         36.4\\
 5 &  8451 &     112.773827 &         792.175312 &                  7 &         37.6\\
\hline
\end{tabular}
\caption{$\kappa = 100$, $\nu_m = 1$ and $\nu=1$}
\end{table}
\end{document}
