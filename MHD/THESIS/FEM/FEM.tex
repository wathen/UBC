\chapter{Finite element discretisation}
\label{sec:discretization}

In this chapter we introduce a mixed finite element discretisation for the steady-state incompressible MHD problem \eqref{eq:mhd}, \eqref{eq:bc} that models electrically conductive fluids under the influence of a magnetic field.  Following the setting in \cite{schotzau2004mixed}, we use curl-conforming elements for the magnetic field and conforming continuous elements for the velocity field. The resulting discretisation is verified though a series of numerical experiments which appear later in Chapter \ref{chap:results}. For simplicity, we initially only discuss in detail homogeneous Dirichlet boundary conditions, that is
\begin{equation} \label{eq:homogeneousBC}
    \uu{u} = \uu{0} \quad \mbox{and} \quad \uu{n}\times \uu{b} = \uu{0}.
\end{equation}
Inhomogeneous conditions as in \eqref{eq:bc}  are  discussed in Section~\ref{sec:bcig}.


\section{Variational formulation}
\label{sec:variation}

\RE{Suppose that the domain $\Omega$ is a Lipschitz domain of $\mathbb{R}^d$ for $d=2,3$.} To express the problem \eqref{eq:mhd}, \eqref{eq:bc} in weak form we follow \cite{schotzau2004mixed} and denote the $L^2$-inner product on $L^2(\Omega)^d$ by $(\cdot,\cdot)_\Omega$, for $d = 2,3$. We introduce the standard Sobolev spaces
\begin{equation} \label{eq:FuncSpace}
 \left. \begin{aligned}
\hspace{-1.5mm}\uu{V}&=H_0^1(\Omega)^d=\left\{\uu{u}\in H^1(\Omega)^d\,:\,\text{$\uu{u}=\uu{0}$ on $\partial\Omega$}\right\},\\
\hspace{-1.5mm}Q&=L^2_0(\Omega)=\{p\in L^2(\Omega)\,:\,(p\,,1)_\Omega=0\},\\
\hspace{-1.5mm}\uu{C}&=H_0({\rm curl};\Omega) = \left\{\uu{b}\in L^2(\Omega)^d\,:\,\nabla\times\uu{b}\in L^2(\Omega)^{\bar{d}}, \
\text{$\uu{n}\times\uu{b}=\uu{0}$ on $\partial\Omega$}\right\},\\
\hspace{-1.5mm}S&=H^1_0(\Omega)=\{r\in H^1(\Omega)\,:\,r=0\ \mbox{on $\partial\Omega$}\}.
 \end{aligned}
 \right.
 \qquad \text{}
\end{equation}
where $\bar{d}={2d-3}$ is used to cover the 2D and 3D cases. We write $\|\cdot\|_{L^2(\Omega)}$, $\|\cdot\|_{H^1(\Omega)}$ and $\|\cdot\|_{H(\rm{curl};\Omega)}$ for the associated natural norms. More precisely, for  vector fields $\uu{u},\uu{b}$ and a scalar function $r$ the norms are defined as follows:
\begin{equation} \nonumber
 \left. \begin{aligned}
    \|\uu{u}\|_{L^2 (\Omega)} &= \left({\int_{\Omega} \uu{u}\cdot\uu{u}\;dx}\right)^{\frac{1}{2}},\\
   \|\uu{u}\|_{H^1(\Omega)} &=  \left(\|\uu{u}\|_{L^2(\Omega)}^2 + \|\nabla  \uu{u}\|_{L^2(\Omega)}^2 \right)^{\frac{1}{2}},\\
   \|\uu{b}\|_{H(\rm{curl},\Omega)} &=  \left(\|\uu{b}\|_{L^2(\Omega)}^2 + \|\nabla \times \uu{b}\|_{L^2(\Omega)}^2 \right)^{\frac{1}{2}}, \\
    \|r\|_{L^2 (\Omega)} &= \left({\int_{\Omega} r^2\;dx}\right)^{\frac{1}{2}},\\
    \|r\|_{H^1(\Omega)} &=  \left(\|r\|_{L^2(\Omega)}^2 + \|\nabla  r\|_{L^2(\Omega)}^2 \right)^{\frac{1}{2}},\\
 \end{aligned}
 \right.
 \qquad \text{}
\end{equation}
where $\|\nabla  \uu{u}\|_{L^2(\Omega)}^2$ is given by:
$$\|\nabla  \uu{u}\|_{L^2(\Omega)}^2 = \left(\int_{\Omega} \sum^d_{i,j=1}(\nabla \uu{u})_{ij}(\nabla \uu{u})_{ij} \, dx\right)^{\frac12}.$$
The weak formulation of the incompressible MHD system (\ref{eq:mhd}), (\ref{eq:bc}) consists in finding~$(\uu{u},p,\uu{b},r)\in \uu{V} \times Q\times \uu{C} \times S$ such that
\begin{subequations}
\label{eq:weak}
\begin{eqnarray}
\label{eq:weak1} A(\uu{u},\uu{v}) + O(\uu{u};\uu{u},\uu{v})
+C(\uu{b};\uu{v},\uu{b})
+B(\uu{v}, p) & =& (\uu{f}, \uu{v})_{\Omega},\\[.1cm]
\label{eq:weak2}
B(\uu{u},q)&=&0, \\[.1cm]
\label{eq:weak3}
M(\uu{b},\uu{c})-C(\uu{b};\uu{u},\uu{c})+D(\uu{c},r)&=& (\uu{g},\uu{c})_\Omega, \\[.1cm]
\label{eq:weak4} D(\uu{b},s)&=&0,
\end{eqnarray}
\end{subequations}
for all $(\uu{v},q,\uu{c},s)\in \uu{V} \times Q\times \uu{C}\times
S$. The individual variational forms are given by
\begin{equation} \label{eq:forms}
 \left. \begin{aligned}
&A(\uu{u},\uu{v})=  \int_\Omega \nu \, \nabla\uu{u}:
\nabla\uu{v}\,d\uu{x},&\\  & O(\uu{w};\uu{u},\uu{v}) = \int_\Omega
(\uu{w}\cdot\nabla)\uu{u} \cdot\uu{v} \, d\uu{x},
\\[.1cm]
&  B(\uu{u},q) = -\int_\Omega\,(\nabla\cdot\uu{u}) \,q \,d\uu{x},
&\\  &
 M(\uu{b},\uu{c})= \int_\Omega\, \kappa\nu_m
(\nabla\times\uu{b})\cdot(\nabla\times\uu{c})\,d\uu{x},\\[0.1cm]
& D(\uu{b},s) = \int_\Omega\, \uu{b} \cdot \nabla s\,
d\uu{x}, & \\ &
C(\uu{d};\uu{v},\uu{b}) =  \int_\Omega \kappa\, (\uu{v}\times\uu{d})\cdot
(\nabla\times\uu{b})\, d\uu{x},
 \end{aligned}
 \right.
 \quad\text{}
\end{equation}
where  $\nabla \uu{u}:\nabla \uu{v}$ is  defined as
$$\nabla \uu{u}:\nabla \uu{v} = \sum^d_{i,j=1}(\nabla \uu{u})_{ij}(\nabla \uu{v})_{ij}.$$ In \cite{schotzau2004mixed} it has been shown that this formulation of the problem is discretely energy-stable and has a unique solution for small data (i.e. for small $\nu$, $\nu_m$, $\kappa$ and forcing terms $\uu{f}$ and $\uu{g}$ with small $L^2$-norms).

\section{Mixed finite element discretisation}

Consider the domain $\Omega$ to be divided up into a regular and quasi-uniform mesh ${\mathcal T}_h=\{K\}$ consisting of triangles ($d = 2$) or tetrahedra ($d = 3$)  with mesh size $h$. Based on the function spaces defined in \eqref{eq:FuncSpace}, our finite element approximation will be sought in the finite dimensional spaces given by:
\begin{equation}
\label{eq:FiniteSpace}
\begin{split}
\uu{V}_h &=  \{\, \uu{u}\in H^1( \Omega)\, :\, \uu{u}|_K \in {\mathcal P}_{k}(K)^d, \, K \in{\mathcal T}_h \, \},\\[.1cm]
Q_h&=  \{\, p\in L^2(\Omega) \cap H^1(\Omega)\,:\, p|_K \in {\mathcal P}_{k-1}(K), \, K \in{\mathcal T}_h \,\},\\[.1cm]
\uu{C}_h &=  \{\, \uu{b}\in H_0({\rm curl}; \Omega) \,:\, \uu{b}|_K \in {\mathcal P}_{k-1}(K)^d \oplus \uu{R}_k(K), \, K \in{\mathcal T}_h \,\},\\[.1cm]
S_h&=  \{\, r\in H_0^1(\Omega) \,:\, r|_K \in {\mathcal P}_{k}(K), \, K \in {\mathcal T}_h \, \},
\end{split}
\end{equation}
for $k\geq 2$. We define ${\mathcal P}_{k}(K)$ as the space of polynomials of total degree at most $k$ on $K$ and $ \uu{R}_k(K)$ as the space of homogeneous vector polynomials of total degree $k$ on $K$ that are orthogonal to the position vector $\uu{x}$. Here we note that we are using ${\mathcal P_k}/{\mathcal P_{k-1}}$ Taylor-Hood elements for the fluid unknowns $(\uu{u},p)$ \cite{taylor1973numerical}. For the magnetic variables $(\uu{b},r)$ we use the curl-conforming \nedelec element pair     of the first kind \cite{nedelec1980mixed}. These choices of finite elements spaces $\uu{V}_h, \, \uu{C}_h, \, Q_h$ and $S_h$ imply that  we have conforming subspaces to our Sobolev spaces $\uu{V}, \, \uu{C}, \,Q$ and $S$, respectively. Then the finite element solution to \eqref{eq:weak} consists in finding $(\uu{u}_h,p_h,\uu{b}_h,r_h)\in \uu{V}_h\times Q_h\times \uu{C}_h\times S_h$ such that
\begin{subequations}
\label{eq:VariationForm}
\begin{eqnarray}
\label{eq:bn1} \hspace{-15mm} A(\uu{u}_h,\uu{v}) + \tilde{O}(\uu{u}_h;\uu{u}_h,\uu{v}) +C(\uu{b}_h;\uu{v},\uu{b}_h) +B(\uu{v}, p_h) & = & ( \uu{f},\uu{v}),\\[.1cm]
\label{eq:bn2}
B(\uu{u}_h,q)&=& 0, \\[.1cm]
\label{eq:bn3} M(\uu{b}_h,\uu{c})-C(\uu{b}_h;\uu{u}_h,\uu{c})+ D(\uu{c},r_h)&=& (\uu{g},\uu{c}),\\[.1cm]
\label{eq:bn4} D(\uu{b}_h,s)&=&0,
\end{eqnarray}
\end{subequations}
for all $(\uu{v},q,\uu{c},s)\in \uu{V}_h\times Q_h \times \uu{C}_h\times S_h$.

The forms $A, M, B, D$ and $C$ stay the same as on the continuous level. However, for the convection term $\tilde{O}(\cdot;\cdot,\cdot)$ we modify the form $O(\uu{w};\uu{u},\uu{v})$ in a standard fashion to ensure the energy-stability property
\begin{equation} \label{eq:convection}
    \tilde{O}(\uu{w};\uu{u},\uu{u}) = 0, \quad \forall \uu{w},\uu{u} \in  \uu{V}_h.
\end{equation}
To do so we integrate by parts the convection form $O(\uu{w};\uu{u},\uu{u})$  to obtain
\begin{equation} \nonumber
 \left. \begin{aligned}
     \int_\Omega (\uu{w}\cdot\nabla)\uu{u} \cdot\uu{u} \, d\uu{x} =& -\frac{1}{2}\int_{\Omega} \nabla \cdot \uu{w} \uu{u} \cdot \uu{u} \, d\uu{x}
     +\frac{1}{2}\int_{\partial \Omega} \uu{w}\cdot \uu{n} |\uu{u}|^2\, ds,
 \end{aligned}
 \right.
 \qquad \text{}
\end{equation}
recalling that $\uu{n}$ is the unit outward normal on $\partial \Omega$. Therefore, we choose the modified convection form $\tilde{O}(\uu{w};\uu{u},\uu{v})$ as
$$\tilde{O}(\uu{w};\uu{u},\uu{v}) =  \int_\Omega (\uu{w}\cdot\nabla)\uu{u} \cdot\uu{v} \, d\uu{x} +\frac{1}{2}\int_{\Omega} \nabla \cdot \uu{w} \uu{u} \cdot \uu{v}\, d\uu{x}-\frac{1}{2}\int_{\partial \Omega} \uu{w}\cdot \uu{n} \uu{u} \cdot \uu{v}\, ds.$$
By construction, property \eqref{eq:convection} is now satisfied. Note also that for homogeneous boundary conditions as assumed in \eqref{eq:homogeneousBC}, the boundary integral term in $\tilde{O}$ can be omitted.

Again in \cite{schotzau2004mixed} it has been shown that this variational formulation of a MHD problem is discretely energy-stable and has a unique solution for small data. Also, optimal order error estimates in the mesh size $h$ have been derived for small data using the stability property \eqref{eq:convection}. Namely, for sufficiently smooth solutions, we have the error bound
$$\|\uu{u}-\uu{u}_h\|_{H^1(\Omega)}+\|\uu{b}-\uu{b}_h\|_{H(\rm{curl};\Omega)}+\|p-p_h\|_{L^2(\Omega)}+\|r-r_h\|_{H^1(\Omega)} \leq C h^k,$$
for a constant $C>0$ independent of the mesh size. In addition, the $L^2$-norm error for the velocity field is of order $\mathcal{O}(h^{k+1})$ (as $\uu{V}_h$ consists of a full polynomial space on each element). However, we cannot expect $L^2$-norm errors of  order $\mathcal{O}(h^{k+1})$ for the magnetic field (as $\uu{C}_h$ does not consist of a full polynomial space on each element).


\subsection{Matrix representation}

The variational formulation \eqref{eq:VariationForm} now can be converted into a matrix representation. To do this, we introduce the basis function for the finite element spaces in \eqref{eq:FiniteSpace}:
\begin{alignat}2
\label{eq:bases1}
\uu{V}_h & = \mbox{span}\langle  \uu{\psi}_j \rangle _{j=1}^{n_u}, & \qquad &
Q_h  = \mbox{span} \langle  \alpha_i \rangle _{i=1}^{m_u},\\[0.1cm]
 \uu{C}_h& =\mbox{span}\langle \uu{\phi}_j \rangle _{j=1}^{n_b}, & \qquad & S_h = \mbox{span} \langle \beta_i
\rangle_{i=1}^{m_b}.
\end{alignat}
The aim now is to find the coefficient vectors $u = (u_1, \ldots , u_{n_u}) \in \mathbb{R}^{n_u}$, $p = (p_1, \ldots , p_{m_u}) \in \mathbb{R}^{m_u}$, $b = (b_1, \ldots , b_{n_b}) \in \mathbb{R}^{n_b}$, and $r = (r_1, \ldots , r_{m_b}) \in \mathbb{R}^{m_b}$ of the finite element functions $(\uu{u}_h, p_h,\uu{b}_h, r_h)$ in terms of the chosen bases. As usual, this is done by writing the bilinear forms in \eqref{eq:VariationForm} in terms of the following stiffness matrices and load vectors:
\begin{alignat*}2
A_{i,j} &= A(\uu{\psi}_j,\uu{\psi}_i), &\quad  &1 \leq i,j \leq n_u,\\[0.1cm]
B_{i,j} &= B(\uu{\psi}_j,\alpha_i), &\quad &1 \leq i \leq m_u, \ 1 \leq j \leq n_u,\\[.1cm]
D_{i,j} &= D(\uu{\phi}_j,\beta_i),  & & 1 \leq i \leq m_b,\ 1 \leq j \leq n_b,\\[.1cm]
M_{i,j}&= M(\uu{\phi}_j,\uu{\phi}_i), &\qquad & 1 \leq i,j \leq n_b,\\[.1cm]
f_i &= (\uu{f},\uu{\psi}_i)_\Omega, & & 1\leq i\leq n_u,\\[.1cm]
g_i &= (\uu{g},\uu{\phi}_i)_\Omega, & & 1\leq i \leq n_b.
\end{alignat*}
For the two non-linear forms, $\tilde{O}$ and $C$, we define the corresponding stiffness matrices with respect to given finite element functions $\uu{w}_h \in \uu{V}_h$ and $\uu{d}_h\in \uu{C}_h$ in the first argument and their associated coefficient vectors $w$ and $d$ as
\begin{alignat*}2
O(w)_{i,j} &=\tilde{O}(\uu{w}_h;\uu{\psi}_j,\uu{\psi}_i), &\quad  &1 \leq i,j \leq n_u,\\[.1cm]
C(d)_{i,j} &= C(\uu{d}_h;\uu{\psi}_j,\uu{\phi}_i), & & 1\leq i \leq n_b,\ 1 \leq j \leq n_u.
\end{alignat*}

Thus, the numerical solution to \eqref{eq:mhd} consists in solving the non-linear system
\begin{equation}
\label{eq:matrix-system}
\left(
\begin{array}{cccc}
A+O(u) & B^T & C^T(b) & 0\\
B & 0 & 0 & 0\\
-C(b) & 0 & M & D^T \\
0 & 0 & D & 0
\end{array}
\right)
\,
\left(
\begin{array}{c}
u\\
p\\
b\\
r
\end{array}
\right) =
\left(
\begin{array}{c} f\\0\\g\\0
\end{array}
\right),
\end{equation}
where the vectors  $u\in\mathbb{R}^{n_u}$, $p\in\mathbb{R}^{m_u}$,  $b\in\mathbb{R}^{n_b}$, and $r\in\mathbb{R}^{m_b}$ are the unknown coefficients of the finite element functions.

\section{Picard iteration (P)}
\label{sec:nonlinear}
The discrete system \eqref{eq:matrix-system} is non-linear, and therefore appling a non-linear solver to this problem is necessary. A common choice to deal with the non-linearity within the incompressible Navier-Stokes equations in isolation is to perform Oseen or Picard iterations \cite{elman2005finite}. This involves linearising around the current velocity and solving for updates.

% One (at least theoretical) advantage of the discrete Oseen system is that it is provably energy-stable (for the skew-symmetrized form which we use) for all values of nu. That is, the matrix is not symmetric but positive definite, which might be an advantage for preconditioning. For small data (i.e. for large \nu) the fixed-point iteration is a contraction.

\RE{For simplicity we only consider the linearly convergent Picard iterations. Since we have modified the convection form to be discretely energy-stable as in \eqref{eq:convection}, an advantage of this approach is that the discrete convection-diffusion operator is real positive. Thus, with small data the fixed-point/Picard iteration is a contraction. A more efficient non-linear solver is Newton's method which converges quadratically near the solution. However, applying Newton's method is more involved, as it requires construction and solving linear systems associated with a Jacobian as well as finding an initial guess sufficiently close to the solution.
% A common approach to ensure convergence with Newton's method is to perform a few Picard iterations before starting the Newton scheme.
% In Section~5.2 we mention the possibility of using Newton's method as an area of future work.
We leave the implementation of Newton's method or other non-linear solvers as an area of possible future work (Section~5.2).
}

% \RE{For simplicity we only consider the linearly convergent Picard iterations. An example of a more efficient non-linear solver is Newton's method which converges quadratically. However, the main difficulty with Newton's method is that convergence is only local and hence the initial guess needs to be sufficiently accurate to obtain convergence. Since we have modified the convection form to be descretely energy-stable \eqref{eq:convection} then an advantage of the discrete Oseen systems is that the matrix is positive definite. Thus, this provides an advantage for preconditioning and with small data the fixed-point/Picard iterations is a contraction.
% % A common approach to ensure convergence with Newton's method is to perform a few Picard iterations before starting the Newton scheme.
% % In Section~5.2 we mention the possibility of using Newton's method as an area of future work.
% We leave the implementation of Newton's method or other non-linear solvers as an area of possible future work (Section~5.2).
% }

\RE{We adapt the fixed-point/Picard iterations to an approach for the full MHD system, where we linearise around the current velocity and magnetic fields.} Given a current iterate $(\uu{u}_h,p_h,\uu{b}_h,r_h)$  we solve for updates $(\delta \uu{u}_h,\delta p_h,\delta \uu{b}_h,\delta r_h)$ and introduce the next iterate by setting:
\begin{equation}\nonumber
\begin{array}{cc}
% \label{eq:updates}
\uu{u}_h& \hspace{-3mm} \rightarrow \uu{u}_h +\delta \uu{u}_h, \quad p_h \rightarrow p_h +\delta p_h,\\
\uu{b}_h& \hspace{-3mm}  \rightarrow \uu{b}_h +\delta \uu{b}_h, \quad r_h \rightarrow r_h +\delta r_h.
\end{array}
\end{equation}
In variational form, the updates $(\delta \uu{u}_h,\delta p_h,\delta \uu{b}_h,\delta r_h)\in \uu{V}_h\times Q_h \times \uu{C}_h\times S_h$ are found by solving the Picard system (P):
\begin{equation} \nonumber
% \label{eq:picard}
\begin{split}
A(\delta\uu{u}_h, \uu{v}) +\tilde{O}(\uu{u};\delta\uu{u}_h,\uu{v})+ C(\uu{b}_h;\uu{v},\delta \uu{u}_h) + B(\uu{v}, \delta p_h) & = R_u(\uu{u}_h,\uu{b}_h,p_h;\uu{v}),\\[.1cm]
B(\delta\uu{u}_h,q)&= R_p(\uu{u}_h;q), \\[.1cm]
M(\delta \uu{b}_h,\uu{c})+
D(\uu{c},\delta r_h)-C(\uu{b}_h;\delta \uu{u}_h,\uu{v})&= R_b(\uu{u}_h,\uu{b}_h,r_h;\uu{c}),\\[.1cm]
D(\delta \uu{b}_h,s)&= R_r(\uu{b}_h;s),
\end{split}
\end{equation}
for all $(\uu{v},q,\uu{c},s)\in \uu{V}_h\times Q_h \times \uu{C}_h\times S_h$. Note that this system is linearised around $(\uu{u}_h,\uu{b}_h)$. The right-hand side linear forms correspond to the residual at the current iteration $(\uu{u}_h,p_h,\uu{b}_h,r_h)$ defined by:
\begin{align*}
 R_u(\uu{u}_h,\uu{b}_h,p_h;\uu{v})&=(\uu{f}, \uu{v})_\Omega-A(\uu{u}_h,\uu{v})
-  \tilde{O}(\uu{u}_h;\uu{u}_h,\uu{v}) \\  & \hspace{4.2mm}- C(\uu{b}_h;\uu{v},\uu{b}_h)-B(\uu{v},p_h),\\[.1cm]
R_p(\uu{u}_h;q)&=-B(\uu{u}_h,q),\\[.1cm]
 R_b(\uu{u}_h,\uu{b}_h,r_h;\uu{c})&=(\uu{g,c})_\Omega -M(\uu{b}_h,\uu{c})
+ C(\uu{b}_h;\uu{u}_h,\uu{c})-D(\uu{c},r_h),\\[.1cm]
R_r(\uu{b}_h;s)&=-D(\uu{b}_h,s),
\end{align*}
for all $(\uu{v},q,\uu{c},s)\in \uu{V}_h\times Q_h \times \uu{C}_h\times S_h$.

In \cite{schotzau2004mixed} it is shown that for small data the Picard iteration (P) will converge to the exact solution for any initial guess.

To formulate the variational form of the Picard iteration (P) in matrix form, let $({u},p,{b},r)$ be the coefficient vectors associated with $(\uu{u}_h,p_h,\uu{b}_h,r_h)$ and $(\delta{u},\delta p,\delta{b},\delta r)$ be the coefficient vectors of $(\delta \uu{u}_h,\delta p_h,\delta \uu{b}_h,\delta r_h)$. Then it can readily seen that the Picard iteration (P) amounts to solving the matrix system
\begin{equation}
\label{eq:mhd_saddle}
%\mathcal{K} x \equiv
\left(
\begin{array}{cccc}
A+O(u) & B^T & C(b)^T & 0\\
B & 0 & 0 & 0 \\
-C(b) & 0 & M & D^T\\
0 & 0 & D & 0
\end{array}
\right)
\,
\left(
\begin{array}{c}
\delta u\\
\delta p\\
\delta b\\
\delta r
\end{array}
\right)  =
\begin{pmatrix}
r_u \\
r_p\\
r_b\\
r_r
\end{pmatrix},
\end{equation}
with
\begin{equation} \label{eq:rhsupdate}
\begin{array}{rl}
r_u &= f- Au -O(u) u - C(b)^T b- B^T p,\\
r_p &=-B u,\\
r_b &=g-Mu+C(b)b-D^T r,\\
r_r &=-D b.
\end{array}
\end{equation}
At each non-linear iteration, the right hand side vectors and matrices $O(u)$ and $C(b)$ in \eqref{eq:rhsupdate} and \eqref{eq:mhd_saddle} respectively must be assembled with the solution coefficient vectors $({u},p,{b},r)$ of the current iterate. Here, the matrix $A$  is symmetric positive definite (SPD), $O(u)$ is non-symmetric and $-C(b)$, $C(b)^T$ appear in a skew symmetric fashion. We also note that $M$ is symmetric positive semidefinite (SPSD) with nullity $m_b$ corresponding to the dimension of the scalar space $S_h$ giving rise to the discrete gradients, see \cite{greif2007preconditioners}.


\section{Decoupled iterations}
\label{sec:FEMdecouple}


The full MHD system \eqref{eq:mhd}, \eqref{eq:bc} is a coupled system consisting of the incompressible Navier-Stokes and Maxwell's equations, coupled through the non-linear skew symmetric coupling term $C(b)$. In addition, the convection term $O(u)$ is non-linear as well. These two terms make the numerical solution challenging. Therefore, if one or both of these terms is small then it may be possible to iterate explicitly. In particular if the coupling term, $C(b)$, is small then we may completely decouple the system into an incompressible Navier-Stokes problem and a Maxwell problem. The two resulting decoupling schemes are what we call Magnetic and Complete Decoupling and are both described below. Note that unlike the Picard iteration, there is no small data guarantee that such iterations based on these decoupling schemes will converge; although we see convergence for reasonable values of the non-dimensional parameters.


\subsection{Magnetic Decoupling (MD)}
\label{sec:FEMmd}

Consider first the situation where there is  weak coupling within the system, that is when $C(b)$ is small. Then it may be possible to drop these terms to completely decouple the system into the two subproblems, the incompressibleNavier-Stokes and Maxwell's equations. We will call this approach Magnetic Decoupling (MD).
% For a given solution $(\uu{u}_h,p_h,\uu{b}_h,r_h)$, neglecting the coupling terms in \eqref{eq:picard} results in solving for the updates $(\delta \uu{u}_h,\delta p_h,\delta \uu{b}_h,\delta r_h) \in \uu{V}_h \times Q_h \times \uu{C}_h \times S_h$  such that
% \begin{equation}
% \label{eq:picard_explicit_MD}
% \begin{split}
% A(\delta\uu{u}_h, \uu{v}) +O(\uu{u};\delta\uu{u}_h,\uu{v})+ B(\uu{v}, \delta p_h) & = R_u(\uu{u}_h,\uu{b}_h,p_h;\uu{v})\\[.1cm]
% B(\delta\uu{u}_h,q)&= R_p(\uu{u}_h;q), \\[.1cm]
% M(\delta \uu{b}_h,\uu{c})+
% D(\uu{c},\delta r_h)&= R_b(\uu{u}_h,\uu{b}_h,r_h;\uu{c}),\\[.1cm]
% D(\delta \uu{b}_h,s)&=R_r(\uu{b}_h;s),
% \end{split}
% \end{equation}
% where again $(\uu{v},q,\uu{c},s)\in\uu{V}_h\times Q_h\times\uu{C}_h\times S_h$ and $R_u$, $R_p$, $R_b$ and $R_r$ which are defined in section \ref{sec:nonlinear}. Again, let $({u},p,{b},r)$ be the coefficient vectors of $(\uu{u}_h,p_h,\uu{b}_h,r_h)$ and $(\delta{u},\delta p,\delta{b},\delta r)$ be the coefficient vectors of $(\delta \uu{u}_h,\delta p_h,\delta \uu{b}_h,\delta r_h)$, then this amounts to solving the linear system:
Then the system\eqref{eq:mhd_saddle} simplifies to
\begin{equation}
\label{eq:matrix_MD}
%\mathcal{K} x \equiv
\left(
\begin{array}{cccc}
A+O(u) & B^T & 0 & 0\\
B & 0 & 0 & 0 \\
0 & 0 & M & D^T\\
0 & 0 & D & 0
\end{array}
\right)
\,
\left(
\begin{array}{c}
\delta u\\
\delta p\\
\delta b\\
\delta r
\end{array}
\right)  =
\begin{pmatrix}
r_u \\
r_p\\
r_b\\
r_r
\end{pmatrix},
\end{equation}
with
\begin{align*}
r_u &= f- Au -O(u)u - C(b)^T b- B^T p,\\[0.1cm]
r_p &=-B u,\\[0.1cm]
r_b &=g-Mu+C(b)b-D^T r,\\[0.1cm]
r_r &=-D b.
\end{align*}
We iterate in the same fashion as the Picard iteration with the simpler matrix \eqref{eq:matrix_MD}. From \eqref{eq:matrix_MD} we can see that the system is now completely decoupled. This enable us to  solve each individual subproblem separately and possibly in parallel.

\subsection{Complete Decoupling (CD)}
\label{sec:FEMcd}

For the second decoupling scheme, we again consider there to be weak coupling of the system but we also consider that the fluid equations are diffusion-dominated and hence can exclude the convection terms.
% This is the simplest technique as it removes all non-linear terms. Again, for a given solution $(\uu{u}_h,p_h,\uu{b}_h,r_h)$, removing the coupling and convection terms in \eqref{eq:picard} results in solving for the updates $(\delta \uu{u}_h,\delta p_h,\delta \uu{b}_h,\delta r_h) \in \uu{V}_h \times Q_h \times \uu{C}_h \times S_h$  such that
% \begin{equation}
% \label{eq:picard_explicit_CD}
% \begin{split}
% A_h(\delta\uu{u}_h, \uu{v}) + B(\uu{v}, \delta p_h) & = R_u(\uu{u}_h,\uu{b}_h.p_h;\uu{v})\\[.1cm]
% B(\delta\uu{u}_h,q)&= R_p(\uu{u}_h;q), \\[.1cm]
% M(\delta \uu{b}_h,\uu{c})+
% D(\uu{c},\delta r_h)&= R_b(\uu{u}_h,\uu{b}_h,r_h;\uu{c}),\\[.1cm]
% D(\delta \uu{b}_h,s)&=R_r(\uu{b}_h;s),
% \end{split}
% \end{equation}
% where $(\uu{v},q,\uu{c},s)\in\uu{V}_h\times Q_h\times\uu{C}_h\times S_h$.  Taking $({u},p,{b},r)$ as the coefficient vectors of $(\uu{u}_h,p_h,\uu{b}_h,r_h)$ and $(\delta{u},\delta p,\delta{b},\delta r)$ be the coefficient vectors of $(\delta \uu{u}_h,\delta p_h,\delta \uu{b}_h,\delta r_h)$, then the proposed decoupled linear system is
This amounts to the system
\begin{equation}
\label{eq:matrix_CD}
%\mathcal{K} x \equiv
\left(
\begin{array}{cccc}
A & B^T & 0 & 0\\
B & 0 & 0 & 0 \\
0 & 0 & M & D^T\\
0 & 0 & D & 0
\end{array}
\right)
\,
\left(
\begin{array}{c}
\delta u\\
\delta p\\
\delta b\\
\delta r
\end{array}
\right)  =
\begin{pmatrix}
r_u \\
r_p\\
r_b\\
r_r
\end{pmatrix},
\end{equation}
with
\begin{align*}
r_u &= f- Au -O(u)u - C(b)^T b- B^T p,\\[0.1cm]
r_p &=-B u,\\[0.1cm]
r_b &=g-Mu+C(b)b-D^T r,\\[0.1cm]
r_r &=-D b.
\end{align*}
Again, we perform iterations in the same fashion as the Picard iteration. This is the simplest technique as it removes all non-linear terms in the iteration matrix and hence leaves the linear Stokes problem in the upper $(1,1)$ block matrix.

\section{Inhomogeneous Dirichlet boundary conditions and initial guess}
\label{sec:bcig}

% In this section, we described the formulation of the MHD system \eqref{eq:mhd}, \eqref{eq:homogeneousBC}, that is with homogeneous Dirichlet boundary conditions. In general, the problems we numerically test have inhomogeneous Dirichlet boundary conditions \eqref{eq:bc}.

When considering inhomogeneous Dirichlet boundary conditions as in \eqref{eq:bc}, we still solve \eqref{eq:mhd_saddle}, \eqref{eq:matrix_MD} and \eqref{eq:matrix_CD} for the solution updates with homogeneous Dirichlet boundary conditions. Therefore, in this approach we must incorporate the inhomogeneous Dirichlet boundary conditions only within the initial guess.

% To start the non-linear iteration schemes given in Section \ref{sec:FEMdecouple} we require an initial guess.

To form a suitable initial guess, we solve the decoupled Stokes problem with the inhomogeneous boundary condition (\ref{eq:bc}a):
\begin{equation} \label{eq:StokesInitial}
\left(
\begin{array}{cc}
A & B^T \\
B & 0 \\
\end{array}
\right)
\left(
\begin{array}{c}
u \\
p \\
\end{array}
\right)=\left(
\begin{array}{c}
f \\
0 \\
\end{array}
\right),
\end{equation}
and then the non-symmetric Maxwell problem the with inhomogeneous boundary conditions (\ref{eq:bc}b), (\ref{eq:bc}c):
\begin{equation} \label{eq:MaxwellInitial}
\left(
\begin{array}{cc}
M -C(u) & D^T \\
D & 0 \\
\end{array}
\right)
\left(
\begin{array}{c}
b \\
r \\
\end{array}
\right)=\left(
\begin{array}{c}
g \\
0 \\
\end{array}
\right).
\end{equation}
Here the term $C(u)$ corresponds the coupling term using $u$ (the initial guess for the velocity field). We expect the inclusion of the coupling term to increase the accuracy of the initial guess because additional information of the problem is used.


% When iteratively solving these two systems, the convergence tolerance for the initial guess is important.  Since the updates will have homogeneous boundary conditions then the initial guess needs to incorporate the inhomogeneous boundary conditions. Consider for example the fluid equations: if the discrete Stokes problem is solved approximately to some tolerance, then the we approximately solve the boundary equations of the form:
% $$1\times u_B = u_D,$$
% where $u_B$ is the coefficients of the solution on the boundary and $u_D$ is a sufficiently accurate boundary data interpolation. That is that the initial guess only starts with boundary values of the same order of accuracy as the initial solve.


The inhomogeneous Dirichlet boundary conditions are incorporated in a standard fashion by suitably modifying the matrix system. The outcome of this procedure is that the boundary data interpolation is only performed for the initial guess. Hence, the iterative solves for the initial guess must be run to a sufficient accuracy to ensure the accuracy of the discrete boundary conditions.

% If the subsequent solution updates (which have homogeneous Dirichlet boundary conditions) are solver accurately, the inaccuracy in the boundary values will remain. For similar reasons, low accuracy of the solution updates will lead to perturbation in the boundary conditions and thus in the whole solution.


% Suppose you solve for the initial guesses $(u_0,p_0,b_0,r_0,)$ one would get an error associated with the boundary $(\delta u_B,\delta p_B,\delta b_B,\delta r_B)$ so that
% \begin{equation} \nonumber
%  \left. \begin{aligned}
%         u_0 &= u+\delta u_B, & \quad&
%         p_0 = p+\delta p_B,\\
%         b_0 &= b+\delta b_B,  &\quad &
%         r_0 = r+\delta r_B.\\
%  \end{aligned}
%  \right.
%  \qquad \text{}
% \end{equation}
% When $(u_0,p_0,b_0,r_0,)$ as the initial guess and then solving for the updates (which have homogeneous Dirichlet boundary conditions) there will still be some error associated with the boundaries  which are $(\delta u_B,\delta p_B,\delta b_B,\delta r_B)$. This therefore means that if the Stokes and Maxwell solves are not done accurately enough then when solving for the updates errors will remain within the boundary conditions. Thus the error norms for the full MHD system will be capped by the accuracy of the Stokes and Maxwell solves for the initial guess.

%%%%%%%%%%%%%%%%%%%%%%%%%%%%%%%%%%%%%%%%%%%%%%%%%%%%%%%%%%%%%%%%%%%%%%%%%%%%%%%%%%%%%%%%%%%%%%%%%%%%%%%%%%%%%%%%%%%%%%%%%%%%%%%%%%%%%%%%%%%%%%%%%%%%%%%%%%%%%%%%%%%%%%%%%%%%%%%%%%%%%%%%%%%%%%%%%%%%%%%%%%%%%%%%%%%%%%%%%%%%%%%%%%%%%%%%%%%%%%%%%%%%%%%%%%%%%%%%%%%%%%%%%%%%%%%%%%%%%%%%%%%%%%%%%%%%%%%%%%%%%%%%%%%%%%%%%%%%%%%%%%%%%%%%%%%%%%%%%%%%%%%%%%%%%%%%%%%%%%%%%%%%%%%%%%%%%%%%%%%%%%%%%%%%%%%%%%%%%%%%%%%%%%%%%%%%%%%%%%%%%%%%%%%%%%%%%%%%%%%%%%%%%%%%%%%%%%%%%%%%%%%%%%%%%%%%%%%%%%%%%%%%%%%%%%%%%%%%%%%%%%%%%%%%%%%%%%%%%%%%%%%%%%%%%%%%%%%%%%%%%%%%%%%%%%%%%%%%%%%%%%%%%%%%%%%%%%%%%%%%%%%%%%%%%%%%%%%%%%%%%%%%%%%%%%%%%%%%%%%%%

% To start the Picard iterations given in Section \ref{sec:nonlinear} we require an initial guess. To form the initial guess, we solve the decoupled Stokes problem
% $$
% \left(
% \begin{array}{cc}
% A & B^T \\
% B & 0 \\
% \end{array}
% \right)
% \left(
% \begin{array}{c}
% u \\
% p \\
% \end{array}
% \right)=\left(
% \begin{array}{c}
% f \\
% 0 \\
% \end{array}
% \right),
% $$
% then the non-symmetric Maxwell problem
% $$\left(
% \begin{array}{cc}
% M -C & D^T \\
% D & 0 \\
% \end{array}
% \right)
% \left(
% \begin{array}{c}
% b \\
% r \\
% \end{array}
% \right)=\left(
% \begin{array}{c}
% g \\
% 0 \\
% \end{array}
% \right).$$
% Here the term $C$ corresponds the the coupling term using $u$ (the initial guess for the velocity field). We use the coupling term within the non-symmetric Maxwell problem as we have the initial guess for the velocity field and incorporating this into the solve for the magnetic and multiplier fields initial guess will hopefully increase the accuracy of the initial guess.


% When solving these two systems, the convergence tolerance which we use is important. In particular, inaccurate solutions cause problems with non-homogeneous boundary conditions. This is because if the matrix system with the boundary conditions applied is not solve to a sufficient accuracy then there will be errors in the boundary data. When we solved for the updates we enforce homogeneous Dirichlet boundary conditions (i.e., zero boundary conditions) on the velocity ($u$), magnetic ($b$) and multiplier ($r$) fields. This therefore means that if the Stokes and Maxwell solves are not done accurately enough then when solving for the updates errors will remain within the boundary conditions. Thus the error norms for the full MHD system will be capped by the accuracy of the Stokes and Maxwell solves for the initial guess.




\section{Summary}

In this chapter we reviewed a mixed finite element approximation to the full MHD system given in \eqref{eq:mhd} and \eqref{eq:bc}. We followed the mixed approach outlined in \cite{schotzau2004mixed} and expressed the MHD system in the matrix form \eqref{eq:mhd_saddle}. Using the Picard iteration  \eqref{eq:mhd_saddle} we introduced two possible decoupling schemes ((MD) and (CD)) which may be simpler to solve for. The performance of the resulting three non-linear iteration schemes depends on the values of the parameters $\kappa$, $\nu$ and $\nu_m$. The next chapter will discuss possible preconditioning approaches for these systems.

In the sequel, we shall omit the dependence of $O(u)$ and $C(b)$ on $u$ and~$b$, respectively, and simply  write $O$ and $C$.








