\chapter{Conclusions and future work}

\section{Conclusions}

In this thesis, we developed an approach for the  numerical solution of an incompressible magnetohydrodynamics (MHD) model, with the goal of solving large scale problems. To that end, we generated a large scale code base, utilising both the finite element software package \fenics \cite{wells2012automated} and linear algebra software {\tt PETSc} \cite{petsc-web-page,petsc-user-ref}.

Our mixed finite element discretisation approach is based on using Taylor-Hood elements for the fluid variables and on a mixed \nedelec pair for the magnetic unknowns. The three linearised iteration strategies for the MHD model range from standard Picard iterations to completely decoupled schemes. For these iteration schemes, we followed the preliminary preconditioning results from \cite{li2010numerical}. The proposed preconditioning ideas are motivated by state of the art preconditioners for the mixed Maxwell and Navier-Stokes subproblems. %$For the Picard scheme we further propose an inner-outer preconditioner.

% In Chapter~2, we presented the mixed finite element formulation used to approximate the MHD model proposed in \cite{schotzau2004mixed}. The approach was based on Taylor-Hood elements for the fluid variables and a \nedelec mixed element pair for the Maxwell problem. Section~\ref{sec:FEMdecouple} introduced two possible decoupling schemes based on the parameters in non-linear  Picard iteration scheme described in Section~\ref{sec:nonlinear}.

% Chapter~3 outlined the preconditioning strategies we employed. First, we considered approaches for the linearised incompressible Navier-Stokes and Maxwell's equations in isolations from \cite{elman2005finite,greif2007preconditioners,MR2911387}. Motivated by the preliminary preconditioning results in \cite{li2010numerical} and the techniques used for the subproblems in isolations, we proposed preconditioning strategies for the linearised systems arising in each of the three non-linear iterations schemes described in  Section~\ref{sec:nonlinear}~and~\ref{sec:FEMdecouple}.

We have introduced two preconditioners for the Navier-Stokes equations. When  applying the Least-Squares Commutator preconditioner we  observed a lack of scalability with respect to the mesh size and the viscosity parameter but for the  Pressure Convection-Diffusion (PCD) preconditioner we obtained scalable results. This  lead us to  using PCD within the preconditioner for the non-linear iteration schemes. The preconditioning results for Maxwell's equations exhibited iteration counts independent of the mesh size and the magnetic viscosity.

% The first part of Chapter~4 (up to and including Section~\ref{sec:Maxwell_validation}), presented convergence and preconditioning results for the linearised incompressible Navier-Stokes and Maxwell's equations in separation. When applying the Least-Squares Commutator preconditioner for the Navier-Stokes equations we a saw lack of scalability with respect to the mesh size and the viscocity parameter but for the  Pressure Convection-Diffusion (PCD) preconditioner we obtain scalable results. This lead us to  using PCD within the the preconditioner for the non-linear iteration schemes. The preconditioning results for Maxwell's equations exhibit iterations independent of the mesh size and the magnetic viscosity.

We have presented results for the three iteration schemes the linearised MHD model. After confirming that the numerical results  meet the expected a priori convergences rates, parameter tests (using direct solvers for the matrix coefficients of the iteration schemes) were carried out to see the performance of the two decoupling schemes as well as the Picard iteration. From the parameter tests we saw that the robustness of the three schemes greatly depends on the non-dimensional parameters in the problem. The Picard iterations behaved  similarly to the Magnetic Decoupling (MD) scheme for relevant parameters based on physical applications. However, the Complete Decoupling (CD) scheme did not converge to the solution for small kinematic viscosities (i.e., $\nu = 0.1$).

Finally, we showed preconditioning results for the MHD linearisations resulting from the three non-linear schemes. One of the principal objectives of this thesis was to perform large scale  preconditioning tests. Here 2D scalable experiments were run with 24 million degrees of freedom for the two decoupled iterations. For the Picard iteration, results were only presented for up to 1.5 million degrees of freedom. We were able to go to much greater problem sizes for the decoupled iterations ((MD) and (CD)) as the preconditioning approach is much less involved (no inner-outer iterations).

We also considered the same parameter tests as above but using the preconditioning approaches; the results examined the robustness of the decoupled preconditioned techniques. Again, the numerical results showed iterations counts independent of the problem size. From the results the convergence of all three iterations ((P), (MD) and (CD)) was hindered by small values of $\nu$. In fact, only the (MD) scheme converged for a $\nu = 0.01$. We stress the need to find more robust preconditioning techniques for small values of the viscosity; see Section~5.2. Similar conclusions can be drawn from the scalable 3D numerical results presented here.

From the preconditioned and non-preconditioned parameter tests in this thesis, it seems that (MD) performs best for feasible physical parameters (small values of $\nu$, $\kappa \approx 1$ and $\nu_m \approx10$). The Picard iteration is significantly more costly due to the inner-outer iterations and hence, should only be used in the ranges where (MD) or (CD) do not converge. If memory and the values of the parameters are moderate (e.g.,  $\nu\geq 1$) then the (CD) scheme is best. This is because the coefficient matrix for (CD) is symmetric and therefore we use the cheaper MINRES Krylov solver.

% \RE{The problems considered in the thesis are idealised models to a large extent, in terms of uniform regular grids, parameter set up, and other aspects. For many practical problems the domains in which the model is discretised may be non-convex. Our approach should work with such domains, but may require adaptive mesh refinement and other non-trivial adjustments. As for solution methods, we have considered problems of sufficiently small size which enabled us to use direct solves on the preconditioners. In practice, however, many real-world 3D problems are very large, and scalable preconditioned iterative solution methods would be required. This could be accomplished by designing specific multigrid methods for the differential operators that arise from the preconditioning techniques. Such specialised methods are generally rather involved and designing them may pose a serious challenge.}

% The scalable results presented in this thesis should allow in-depth studies into MHD applications.


\RE{The experiments carried out in the thesis are quite idealised.
We employed uniform regular grids, and tested problems for smooth solutions with relatively harmless parameter values. For many practical applications the problems may involve complex 3D geometries with varying, discontinuous or even non-linear material coefficients. In addition, MHD flows may be convection-dominated. Tackling these difficulties in a computationally efficient manner may require additional computational tools such as automatic mesh generation in complicated 3D domains, adaptive mesh refinement techniques, as well as refined finite element models to handle strong convection and more complex material laws.}

\RE{As for solution methods, in this thesis we have considered problems of sufficiently small size which enabled us to use direct solves on the preconditioners. In practice, however, large real-world 3D problems require fully scalable preconditioned iterative solution methods, e.g., specialised multigrid techniques}


\RE{The scalable results presented in this thesis provide a starting point for further in-depth studies into these issues in the context of MHD applications.}

% One of the principal objectives of this thesis was to numerically test large scale preconditioning approaches.
% One of the main issues with the 3D code is that the application of the preconditioners become very expensive. This is where an iterative (from multigrid and/or iterative solvers) application of the preconditioner would allow larger problems to be solved.

\section{Future work}

We finish this thesis with some considerations of possible areas for future work.

\begin{itemize}
    \item[\textbf{1.}] \textbf{Scalable inner solvers:} ~\\ In this thesis, the application of the preconditioners have been carried out using direct solves for the component blocks. However, to construct fully scalable preconditioners we require an iterative approach to solve these systems of equations. For the most part, this can be readily achieved by using standard algebraic or geometric multigrid cycles and/or iterative methods. We note though that the scalable iterative solution for the shifted curl-curl operator in \eqref{eq:maxwell_pc_X} is more involved. One option is to employ the multigrid solver developed in \cite{hiptmair2007nodal}. We expect the  implementation of scalable inner solves to significantly improve the efficiency of the overall approach, especially for large scale 3D problems and we should obtain a fully scalable iterative solver.
    \item[\textbf{2.}] \textbf{Release code on a public repository:} ~\\ Once the code has been cleaned up and properly commented, the idea is to release  it for public use on either GitHub \url{https://github.com/} or Bitbucket \url{https://bitbucket.org/}.
    \item[\textbf{3.}] \textbf{Parallelisation of the code:} ~\\ The code to discretise and solve the MHD system \eqref{eq:mhd}, \eqref{eq:bc} is currently executed sequentially. For 3D problems of large dimension, an efficient parallel implementation is expected to greatly decrease setup and solve time, in particular for the decoupling schemes.
    \item[\textbf{4.}] \textbf{Robustness with respect to kinematic viscosity:} ~\\ The results presented in Chapter~4 indicate the  efficient performance with respect to the mesh size and with respect to the non-dimensional parameters ($\nu$, $\nu_m$ and $\kappa$). However, for small fluid viscosities, the performance of all the proposed  preconditioners for our  discretisation starts to degrade. The development of preconditioners and convection stabilised discretisations that work well together and are more robust with respect to $\nu$ is another area of possible future work. The solution of convection-dominated flow is very challenging and is a subject of active research in many other areas.
    \RE{\item[\textbf{5.}] \textbf{High frequencies in Maxwell's equations:} ~\\  The Maxwell formulation considered in this MHD model only considers low frequencies. The consideration of high frequencies introduces hyperbolicity into the system (for the magnetic field $\uu{b}$). It may still be possible to use the proposed discretisation in space, but the time discretisation and keeping the consistency of the previously made assumptions (e.g., non-relativistic motion) may introduce additional challenges.}
    \item[\textbf{6.}] \textbf{Other non-linear solvers:} ~\\ The non-linear solver currently used is  simply the linearised Oseen form (i.e., a linearly convergent Picard iteration) for the MHD system. A possible next step includes considering other non-linear solvers that have faster convergence properties. One such scheme is the Newton iteration, which converges quadratically for a sufficiently accurate initial guess. One challenge here is that the convergence is local, and it is necessary to use various means to compute initial iterates that are close enough to the solution. A possible approach is to start with a few Picard iterations then use the Newton iterations.
    \item[\textbf{7.}] \textbf{Different mixed finite element discretisations:} ~\\ We used Taylor-Hood elements \cite{taylor1973numerical} for the fluid variables. This choice of elements is a very common option when considering mixed finite element discretisations of fluid flow equations. However, when using continuous pressure elements additional errors may arise due to poor mass conservation on the discrete level. A possible example of this phenomenon has been studied in \cite{MR2571343} in the context of colliding flows. One option to overcome this difficulty is to use exactly divergence-free elements such as those proposed in \cite{MR2304270,GreifLiSchotzauWei2010}. The elements there are based on using divergence-conforming elements for the velocities and on discontinuous pressure elements. A discontinuous Galerkin approach is employed to ensure  $H^1$-continuity of the velocity fields. The preconditioning approaches in this thesis are based \cite{li2010numerical}, where exactly divergence-free elements have been used for the fluid unknowns to approximate MHD problem \eqref{eq:mhd}. One major challenge  on the linear algebra front is the development of a scalable iterative solver for such elements. A large-scale study of exactly divergence-free elements for the fluid unknowns, and more generally the investigation of other mixed finite element approximations with discontinuous pressures for  MHD models is another area of possible future work.

\end{itemize}


