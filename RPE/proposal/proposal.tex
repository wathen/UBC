\documentclass[12pt]{article}


\usepackage{amsmath}
\usepackage{amsfonts}
\usepackage{graphicx}
\usepackage{nicefrac}
\usepackage{subfigure}
\usepackage{paralist}
% \usepackage[geometry]{ifsym}
\usepackage{rotating}
\usepackage[normalem]{ulem}
\usepackage{cite}
\usepackage{nicefrac}
\usepackage{varwidth}
% \defbibheading{bibliography}[\refname]{}
% \usepackage[T1]{fontenc}
% \usepackage[utf8]{inputenc}
% \usepackage{mathptmx}
\usepackage{times}
 \usepackage{pslatex}
\usepackage[unicode=true,
  linktocpage,
  linkbordercolor={0.5 0.5 1},
  citebordercolor={0.5 1 0.5},
  linkcolor=blue]{hyperref}

\bibliographystyle{amsplain}

% \addtolength{\oddsidemargin}{-.2in}
%     \addtolength{\evensidemargin}{-.5in}
%     \addtolength{\textwidth}{0.5in}

%     \addtolength{\topmargin}{-0.5in}
%     \addtolength{\textheight}{0.35in}

\sloppy                 % makes TeX less fussy about line breaking

\pagestyle{plain}           % use just a plain page number

\numberwithin{equation}{section}    % add the section number to the equation label


%\usepackage[paperwidth=216mm, paperheight=279mm, margin=2.5cm]{geometry}
\usepackage{fancyheadings}

\newcommand{\com}[1]{\texttt{#1}}
\newcommand{\DIV}{\ensuremath{\mathop{\mathbf{DIV}}}}
\newcommand{\GRAD}{\ensuremath{\mathop{\mathbf{GRAD}}}}
\newcommand{\CURL}{\ensuremath{\mathop{\mathbf{CURL}}}}
\newcommand{\CURLt}{\ensuremath{\mathop{\overline{\mathbf{CURL}}}}}
\newcommand{\nullspace}{\ensuremath{\mathop{\mathrm{null}}}}


\newcommand{\FrameboxA}[2][]{#2}
\newcommand{\Framebox}[1][]{\FrameboxA}
\newcommand{\Fbox}[1]{#1}

%\usepackage[round]{natbib}

\newcommand{\half}{\mbox{\small \(\frac{1}{2}\)}}
\newcommand{\hf}{{\frac 12}}
\newcommand {\HH}  { {\bf H} }
\newcommand{\hH}{\widehat{H}}
\newcommand{\hL}{\widehat{L}}
\newcommand{\bmath}[1]{\mbox{\bf #1}}
\newcommand{\hhat}[1]{\stackrel{\scriptstyle \wedge}{#1}}
\newcommand{\R}{{\rm I\!R}}
\newcommand {\D} {{\vec{D}}}
\newcommand {\sg}{{\hsigma}}
%\renewcommand{\vec}[1]{\ensuremath{\mathbf{#1}}}
\newcommand{\E}{\vec{E}}
\renewcommand{\H}{\vec{H}}
\newcommand{\J}{\vec{J}}
\newcommand{\dd}{d^{\rm obs}}
\newcommand{\F}{\vec{F}}
% \newcommand{\C}{\vec{C}}
\newcommand{\s}{\vec{s}}
\newcommand{\N}{\vec{N}}
\newcommand{\M}{\vec{M}}
\newcommand{\A}{\vec{A}}
\newcommand{\B}{\vec{B}}
\newcommand{\w}{\vec{w}}
\newcommand{\nn}{\vec{n}}
\newcommand{\cA}{{\cal A}}
\newcommand{\cQ}{{\cal Q}}
\newcommand{\cR}{{\cal R}}
\newcommand{\cG}{{\cal G}}
\newcommand{\cW}{{\cal W}}
\newcommand{\hsig}{\hat \sigma}
\newcommand{\hJ}{\hat \J}
\newcommand{\hbeta}{\widehat \beta}
\newcommand{\lam}{\lambda}
\newcommand{\dt}{\delta t}
\newcommand{\kp}{\kappa}
\newcommand {\lag} { {\cal L}}
\newcommand{\zero}{\vec{0}}
\newcommand{\Hr}{H_{red}}
\newcommand{\Mr}{M_{red}}
\newcommand{\mr}{m_{ref}}
\newcommand{\thet}{\ensuremath{\mbox{\boldmath $\theta$}}}
\newcommand{\curl}{\ensuremath{\nabla\times\,}}
\renewcommand{\div}{\nabla\cdot\,}
\newcommand{\grad}{\ensuremath{\nabla}}
\newcommand{\dm}{\delta m}
\newcommand{\gradh}{\ensuremath{\nabla}_h}
\newcommand{\divh}{\nabla_h\cdot\,}
\newcommand{\curlh}{\ensuremath{\nabla_h\times\,}}
\newcommand{\curlht}{\ensuremath{\nabla_h^T\times\,}}
\newcommand{\Q}{\vec{Q}}
\renewcommand{\J}{\vec J}
\renewcommand{\J}{\vec J}
% \newcommand{\U}{\vec u}
\newcommand{\nedelec}{N\'{e}d\'{e}lec }
\newcommand{\Bt}{B^{\mbox{\tiny{T}}}}
\newcommand{\me}{Maxwell's equations }
\newcommand{\ns}{Navier-Stokes Equations }
\renewcommand{\s}{Stokes Equations }
\newcommand{\Fs}{\vec{f}_{\mbox{\tiny s}}}
\newcommand{\partialt}[1]{\frac{\partial #1}{\partial t}}
\newcommand{\cref}[1]{(\ref{#1})}
% \newcommand{\Ct}{\ensuremath{C^{\mbox{\tiny{T}}}}
\newcommand{\Ct}{\ensuremath{C^{\mbox{\tiny{T}}}}}
% \renewcommand{\baselinestretch}{1.40}\normalsize
\usepackage{setspace}
\usepackage{amsthm}
\newtheorem{prop}{Proposition}[section]

% \onehalfspacing
\begin{document}
\pagestyle{fancyplain}
\fancyhead{}
\fancyfoot{} % clear all footer fields
\fancyfoot[LE,RO]{\thepage \hspace{-5mm}}
\rhead{ \footnotesize{ Michael Wathen 7830121}}
\fancyfoot[CO,RE]{}

% \title
\noindent{ \bf Project proposal}

\bigskip


% \author{Michael Wathen}
% \date{\vspace{-10ex}}
%\date{}  % Toggle commenting to test
% \maketitle

This proposal concerns the development of fully scalable and efficient solution techniques for a magnetohydrodynamics (MHD) problem. MHD models describes a coupling between electromagnetic effects and fluid flow governed by the  Maxwell’s equations and Navier-Stokes equations respectively. These problems appear frequently in many applications in science and engineering, and any progress in the development of solution methods has the potential of making a high impact.

\smallskip

Using my Master's degree work (at UBC) as a starting point, we will aim to develop a fully scalable iterative solution method, that is, a solver with optimal complexity. To that end, specialised multigrid methods will be implemented, which are tailored to the specific differential operators arising in the problem. One option is to employ and extend the multigrid solver developed in \cite{hiptmair2007nodal} {which is known to be very robust}. The implementation of scalable preconditioner solves has the potential of significantly improving the efficiency of the solution procedure, especially for large scale three-dimensional problems.
\smallskip


Currently, there is a very limited set of available iterative solution techniques for the MHD model, in part due to enormous computational effort that is required. The availability of increased computing resources and parallel computational platforms has been instrumental in generating new possibilities and horizons in this regard.

\smallskip

Our proposed discretization and linearization, results in a $4\times4$ non-symmetric block-structured linear system needs to be (repetitively) solved. One of the principal challenges is the presence of a skew-symmetric term that couples the fluid velocity with the electric field. The proposed techniques  exploit the block structure of the underlying linear system, utilizing and combining effective preconditioners for the mixed Maxwell and Navier-Stokes subproblems. The preconditioner is based on dual and primal Schur complement approximations to yield a scalable solution method. 

\smallskip

An aim of this work is to provide  engineers, geophysicists and others with the capability of performing computations on specific large scale applications they work on. As part of the project, we will generate a customisable and flexible, publicly available software package.
\smallskip

As an integral part of the research project, we will also consider iterative solution techniques for the Helmholtz equations. These equations are a subclass of electromagnetics problems, which give rise to indefinite linear systems. These are notoriously difficult problems, which are typically ill-posed and hence cause ill-conditioning of the discrete systems. Iterative solution methods are typically limited in their ability to deal with high frequencies, and often require the design of complicated solvers with fine meshes (which generally lead to large linear systems that are hard to solve). Any development of efficient solvers for Helmholtz equations will help the advancement of many diverse applications, such as radar and sonar technologies, noise scattering and seismology. Our plan is to design and implement a preconditioner based on hierarchical semi-separable (HSS) matrices. This is a recursive technique based on block decompositions and repeated rank-revealing factorisations of the off-diagonal blocks \cite{wang2011efficient}, which has been used primarily as a direct solution technique, but to date has not been fully explored as a paradigm for preconditioning.

\smallskip

By the end of project the plan is to release all code onto a public repository such as GitHub \url{https://github.com/}. We believe that making the code public will help increase the development of MHD and Helmholtz solvers for real physical applications. It will allow for combining together several prominent software packages (namely {\tt FEniCS}, {\tt PETSc}, and {\tt hypre}), to solve significant physical problems described by partial differential equations (PDEs).

% \medskip


\renewcommand{\section}[2]{}
\bibliographystyle{apalike}
\bibliography{ref}
\newpage
% \title{\Large Contributions and Statement}
% % \author{Michael Wathen}
% \date{\vspace{-10ex}}
% %\date{}  % Toggle commenting to test
% \maketitle
%
%\noindent{ \bf Contributions and Statement}
%
%\bigskip
%
%
%\noindent{\bf  Part I - Contributions to research and development}
%\medskip
%
%
%\noindent{\bf d. Non-refereed contributions (e.g., specialized publications, technical reports, conference presentations, posters)}
%\begin{itemize}
%    \item Greif, C., Li, D., Sch{\"o}tzau, D., and {\bf Wathen, M.} (2014) Block Preconditioners for Saddle-Point Linear Systems With Focus on Incompressible MHD, Chicago. July 9, 2014, SIAM Annual Meeting. (International conference presentation)
%    \item {\bf Wathen, M.} (2014) Iterative Solution of a Mixed Finite Element Discretisation of an Incompressible Magnetohydrodynamics Problem, UBC Vancouver. (MSc thesis)
%    \item {\bf Wathen, M.} (2014) Publicly available software development to be released.
%    \item Greif, C., Li, D., Sch{\"o}tzau, D., and {\bf Wathen, M.} (2014) Preconditioners for a Mixed Finite Element Disctretization of an Incompressible MHD equations (20 pages complete to date)
%\end{itemize}
%
%
%
%\noindent{\bf  Part II - Most significant contributions to research and development }
%\medskip
%
%To date my most significant contribution to research is my M.Sc work on a scalable iterative solution method for the MHD problem, and in particular, the development and testing of suitable preconditioners. This work has been presented this year at the SIAM Annual Meeting in Chicago, where the main body of the numerical results came directly from my MSc thesis. This thesis was written in a similar style to a journal paper and it is currently being converted into a journal paper with my supervisors Prof. Greif and Prof. Sch{\"o}tzau. The thesis details the non-trivial MHD problem and outlines the challenges and methods that have been attempted before, as well it contributes a computational investigation of possible methods and solution techniques and examines how they perform in a general-purpose setting.
%\smallskip
%% style issue with journal paper twice
%
%One of the considerable new developments from my thesis is the formulation of the MHD model and hence the descretisation we used. For the fluid varibles we used standard finite elements, however, the discretisation used for the magnetic variables (arising from the Maxwell part of the MHD model) was a mixed \nedelec pair. We chose these elements since for non-convex domains (which in general applications have)  \nedelec elements correctly capture the solution at the singular points. The development of preconditioners for this new formulation is my most significant contribution to research.
%\smallskip
%
%Due to the mathematical nature of my M.Sc thesis and the lengthy review process for a paper submitted to a journal rather than a conference, to date the work is unpublished.
%\smallskip
%
%% During my first year of my Masters I took a course from Prof. Eldad Haber in Earth and Ocean Science. For the course project  $-->$ non-standard approach.....
%
%Through my M.Sc work I have been developing a large scale modular and extremely flexible software package to solve the MHD model. The core libraries that make up this package are the finite element software package {\tt FEniCS} and the linear algebra software {\tt PETSc}. This contribution to software development and high performance computing demonstrates how important physical problems can be solved using well known software.
%% Throughout the development of this package
%% Combining these software packages shows how important physical problems may be handled
%\smallskip
%
%In the summer of my second year of my undergrad I worked for Numerical Algorithms Group (NAG Ltd), a well-known producer of high quality mathematical software. My job entailed writing example programs in {\tt{Matlab}} that used the NAG toolbox for {\tt{Matlab}} and creating graphical outputs from NAG routines. My work at NAG appeared in Mark 23 of the NAG toolbox for {\tt{Matlab}}.
%
%
%
%\bigskip
%% check the apostrophe
%\noindent{\bf  Part III - Applicant’s statement}
%\medskip
%
%Since the start of my undergraduate studies at the University of Birmingham, UK, I have had several opportunities to work on fascinating research projects as well as writing code to be used in many important areas. These experiences have developed into a deep interest in high performance computing, particularly in the area of numerical solution of PDEs.
%\smallskip
%
%In the summer of 2011 I was a recipient of the Visiting Research Scholar Award 2011 with Professor Nilima Nigam at Simon Fraser University (SFU) for 10 weeks as an international student. This 10 week project enabled me to gain first hand experience to discover what graduate studies and research might be like. Over this period, I investigated and produced numerical codes in {\tt{Matlab}} to solve a diffusive PDE using Spectral Methods in space together with Exponential Time Differencing. With the computer programs that I wrote I computed results in 1, 2 and 3 space dimensions. I wrote a report in \LaTeX \ on these results and findings. It was this experience of research that led me to feel that I had the skills and enthusiasm for graduate study in computational and applied mathematics.
%\smallskip
%
%For my final year undergraduate project at Birmingham, I chose a topic in which I had to write code and compute numerical solutions to certain PDEs describing flows of thin fluid sheets. In this I used a Stream Surface method and the Chebyshev Tau method for applying boundary conditions. The strong first class standard of my final year marks from the courses and this project allowed me to get gain a place for an MSc in Computer Science at The University of British Columbia (UBC) where I have been concentrating on Scientific Computing for the past two years.
%\smallskip
%
%Since arriving at UBC to start my Master's program, I have been lucky enough to follow advanced courses from Uri Ascher, Chen Greif and Michael Friedlander, all Professors in Computer Science, from Eldad Haber a professor in Earth Sciences and from Anthony Peirce in Mathematics. I have also gained first hand teaching experience in my role as a Teaching Assistant. My duties varied from giving {\tt{Matlab}} tutorials to both undergraduates and graduate students at UBC to holding student office hours.
%\smallskip
%
%All of these experiences have helped to broaden my range and depth of knowledge, strengthened my teaching skills and reinforced the fact that I would like to pursue a career in research.
%

\end{document}
